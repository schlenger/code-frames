\documentclass{article}

% Character encoding
\usepackage[ansinew]{inputenc} % windows
%\usepackage[latin1]{inputenc} % linux
%\usepackage[applemac]{inputenc} % macOS
\usepackage[T1]{fontenc}
\usepackage[hidelinks]{hyperref} % enable links, but down mark them in an ugly manner

% Enable Colors and set ToDos with \todo{text}
\usepackage{color}
\newcommand{\todo}[1]{\textcolor{red}{ToDo: #1}}
% Prints a gray lorem ipsum example text
\newcommand{\lorem}{\textcolor{gray}{Lorem ipsum dolor sit amet, consectetur adipisicing elit, sed do eiusmod
tempor incididunt ut labore et dolore magna aliqua. Ut enim ad minim veniam,
quis nostrud exercitation ullamco laboris nisi ut aliquip ex ea commodo
consequat. Duis aute irure dolor in reprehenderit in voluptate velit esse
cillum dolore eu fugiat nulla pariatur. Excepteur sint occaecat cupidatat non
proident, sunt in culpa qui officia deserunt mollit anim id est laborum.}}

% Enable images
\usepackage[pdftex]{graphicx}          
\graphicspath{{./images/}} % base image path
% base file extension for shorter \includegraphics
\DeclareGraphicsExtensions{.pdf,.jpeg,.png,.jpg} 

% Multi-line comments
\newcommand{\comment}[1]{}

% Title information
\title{Sample Tex File}
\author{Max Mustermann} 

\begin{document}
\maketitle

\tableofcontents

\paragraph{Abstract.}
This is some abstract text. It contains information about the paper and should be around 200�250 words.
\newpage


\section{Sample Section} % (fold)
\label{sec:sample_section}

A basic article.

\subsection{Sample Subsection} % (fold)
\label{sub:sample_subsection}
With an Uml�ut \cite{sample}. And it is shown in figure \ref{sample_figure}, which is in subsection \ref{sub:sample_subsection}.

\begin{figure}[h!]
  \centering
  \includegraphics[width=\textwidth]{sample_image} 
  \caption{A sample figure}
  \label{sample_figure}
\end{figure}
% subsection sample_subsection (end)

% section sample_section (end)


% Bibliography
\begin{thebibliography}{widestlabel}
  \bibitem{sample} An author. "A Book." A conference, 2016. Some publishing information.
\end{thebibliography}

\end{document}